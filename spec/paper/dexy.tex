\documentclass{article}   % list options between brackets

\usepackage{color}
\usepackage{graphicx}
%% The amssymb package provides various useful mathematical symbols
\usepackage{amssymb}
%% The amsthm package provides extended theorem environments
%\usepackage{amsthm}
\usepackage{amsmath}

\usepackage{listings}

\usepackage{hyperref}

\usepackage{systeme}

\def\shownotes{1}
\def\notesinmargins{0}

\ifnum\shownotes=1
\ifnum\notesinmargins=1
\newcommand{\authnote}[2]{\marginpar{\parbox{\marginparwidth}{\tiny %
  \textsf{#1 {\textcolor{blue}{notes: #2}}}}}%
  \textcolor{blue}{\textbf{\dag}}}
\else
\newcommand{\authnote}[2]{
  \textsf{#1 \textcolor{blue}{: #2}}}
\fi
\else
\newcommand{\authnote}[2]{}
\fi

\newcommand{\knote}[1]{{\authnote{\textcolor{green}{Alex notes}}{#1}}}
\newcommand{\mnote}[1]{{\authnote{\textcolor{red}{Amitabh notes}}{#1}}}

\usepackage[dvipsnames]{xcolor}
\usepackage[colorinlistoftodos,prependcaption,textsize=tiny]{todonotes}

%\newcommand{\bruno}[1]{\todo[linecolor=OliveGreen,backgroundcolor=OliveGreen!25,bordercolor=OliveGreen]{#1}}

% type user-defined commands here
\usepackage[T1]{fontenc}

\usepackage{flushend}

%\usepackage{xcolor}

\newcommand{\bc}{ERG}
\newcommand{\stc}{stablecoin}
\newcommand{\sct}{stablecoin}
\newcommand{\dx}{Dexy}

\newcommand{\ma}{\mathcal{A}}
\newcommand{\mb}{\mathcal{B}}
\newcommand{\he}{\hat{e}}
\newcommand{\sr}{\stackrel}
\newcommand{\ra}{\rightarrow}
\newcommand{\la}{\leftarrow}
\newcommand{\state}{state}

\newcommand{\ignore}[1]{} 
\newcommand{\full}[1]{}
\newcommand{\notfull}[1]{#1}
\newcommand{\rand}{\stackrel{R}{\leftarrow}}
\newcommand{\mypar}[1]{\smallskip\noindent\textbf{#1.}}

\begin{document}

\title{Dexy - A Stablecoin With Algorithmic Bank Interventions}
\author{Alexander Chepurnoy, Amitabh Saxena}


\maketitle

\begin{abstract}
In this paper, we consider a new stablecoin design called \em{Dexy}.
\end{abstract}

% % !TEX root = laws.tex

\section{Introduction}

Algorithmic stablecoins is a natural extension of cryptocurrencies, trying to 
solve problems with volatility of their prices by pegging stablecoin price to an
asset which price believed to be more or less stable with time (e.g. 1 gram of gold).  


Stable pricing could be useful for:
\begin{enumerate}
\item securing fundraising;
\item doing business with predictable results. For example, a hosting provider knows 
\item shorting: .
\item lending and other decentralized finance applications. 
\end{enumerate}

Algorithmic stablecoins are different from centralized stablecoins, such as USDT and USDC, which are 
convertible into underlying (pegged asset) via a trusted party. In case of an algorithmic stablecoin, its 
pegging is done via rebasement of total supply, or via imitating the trusted party, which holds. 


\section{Introduction}

Algorithmic stablecoins is a natural extension of cryptocurrencies, trying to 
solve problems with volatility of their prices by pegging stablecoin price to an
asset which price believed to be more or less stable with time (e.g. 1 gram of gold).  

Stable pricing could be useful for:
\begin{enumerate}
\item securing fundraising; a project can be sure that funds collected during fundraising will have stable value in the mid- and long-term.
\item doing business with predictable results. For example, a hosting provider can be sure that payments collected at the beginning of the month 
      would be about the same at the end of the month when the provider will need to pay the bills. 
\item shorting: when cryptocurrency prices are high, it is desirable for investors to rebalance their portfolio by increasing exposure to fiat currencies (or traditional
 commodities). However, as KYC procedures are often cumbersome or not possible, it would be better to buy fiat and commodity substitues in form of stablecoins on decentralized exchanges.
\item lending and other decentralized finance applications. Stability of asset value is critical for many applications.
\end{enumerate}

Algorithmic stablecoins are different from centralized stablecoins, such as USDT and USDC, which are 
convertible into underlying (pegged asset) via a trusted party. In case of an algorithmic stablecoin, its 
pegging is done via rebasement of total supply, or via imitating the trusted party, which holds. 


\section{General Design}

Unlike popular algorithmic stablecoins based on two tokens (instruments), \dx{} is based on one token but two protocols. In the first place, 
there is market where trading of \dx{} vs the base currency (\bc{}) happens. In the second place, if market price is way too different from target price (reported by an oracle), there is an algorithmic central bank which makes interventions. The central bank can also mint new \dx{} tokens, selling them for \bc{}. The bank is using reserves in \bc{} it is buying for interventions then. 

As a simple solution for the market, we are using constant-factor Automated Market Maker (CF-AMM) liquidity pool, similar to ErgoDEX and UniSwap. The pool has \bc{} on one side and \dx{} on another. For CF-AMM pool, multiplication of \bc{} and \dx{} amounts before and after a trade should be preserved, so $e * u = e' * u'$, where $e$ and $u$ are amounts of \bc{} and \dx{} in the pool before the trade, and $e'$ and $u'$ are amounts of \bc{} and \dx{} in the pool after the trade. It is also possible to put liquidity into the pool, and remove liquidity from it. 

The bank has two basic operations. It can mint new \dx{} tokens and 

Now we are going to consider how to put restrictions and design rules for the system to have stable price of \dx{}. 

\section{Stability}

What provides stability for the stablecoin when we have the design sketched in the previous section? 

\section{Worst Scenario and Bank Reserves}

The bank is doing interventions when the situation is far from normal on the markets, and enough time passed for markets to stabilize themselves with no interventions. In our case, the bank is doing interventions based on stablecoin price in the liquidity pool in comparison with oracle provided price~(we can assume that price on other markets is similar to liquidity pool\'s due to arbitrage). The bank\'s intervention then is about injecting its \bc{} reserves into the pool.  

We start with introducing notation: 
\begin{itemize}
  \item{} $T$ - period before intervention starts. After one intervention the bank can start another one after $T$. 
  \item{} $p$ - price reported by the oracle at the moment (for example, 20 USD per ERG)
  \item{} $s$ - price which the bank should stand in case of price crash. For example, we can assume that $s = \frac{p}{4}$ (so if p is 20 USD per ERG, then $s$ is 5 USD per ERG, means the bank needs to have enough reserves to save the markets when the price is crashing from 20 to 5 USD per ERG)
  \item{} $R$ - ratio between $p$ and $s$, $R = \frac{p}{s}$
  \item{} $e$ - amount of \bc{} in the liquidity pool 
  \item{} $u$ - amount of \stc{} in the liquidity pool
  \item{} $O$ - amount of \stc{} outside the liquidity pool. The distribution in $O$ is not known for the \dx{} protocol.
  \item{} $E$ - amount of \bc{} in the bank. 
  \item{} $r$ - ratio between $p$, and price in the pool, which is $\frac{u}{e}$, thus $r = \frac{p}{\frac{u}{e}} = \frac{p*e}{u}$
\end{itemize}  

Note that current price in the pool is $\frac{u}{e}$. 

First of all, let's assume that price crashed from $p$ to $s$ sharply and stands there, and before the crash there were $e$ of \bc{} and $u$ of \sct{}, respectively, with liquidity pool price being $p$. The worst case is when no liqudity put into the pool during the period $T$. With large enough $T$ and large enough $R$ this assumption is not very realistic probably: liquidity will be put into the pool by arbitrage players, price is failing with swings where traders will mint \sct{} by putting \bc{} into bank reserves, and so on. However, it would be reasonable to consider worst-case scenario, then in the real world \dx{} will be even more durable than in theory. 

In this case, the bank must intervene after $T$ units of time, as the price differs significantly, and restore the price in the pool, so set it to $s$. We denote amounts of \bc{} and \sct{} in the pool after the intervention as $e'$ and $u'$, respectively. Then:

\begin{itemize}
  \item{} $e * u = e' * u'$
  \item{} as the bank injects $E_1$ ergs into the pool, $e' = e + E_1$
  \item{} $\frac{u'}{e'} = s$, thus $u' = s * (e + E_1)$ 
  \item{} from above, $E_1 = \sqrt{\frac{e * u}{s}} - e$
\end{itemize}

So by injecting $E_1$ \bc{}, the bank recovers the price. However, this is not enough, as now there are $O$ \sct{} units which can be injected into the pool from outside. 
Again, in the real life it is not realistic to assume that all the $O$ \sct{} would be injected, as some of them are simply lost. However, we need to assume worst-case scenario. We also assume that those $O$ tokens are being sold in very small batches not significantly affecting price in the pool, and after each batch seller of a new batch is waiting for a bank intervention to happen (so for $T$ units of time), and sells only after the intervention. In this case all the $O$ tokens are being sold at price close to $s$, so the bank should have $E_2 = \frac{O}{s}$. We note that this scenario is also not realistic and takes very long time. However, as before we assume the absolutely worst case.

Summing up $E_1$ and $E_2$, we got \bc{} reserves the bank should have to be ready for worst-case scenario: $E_w = E_1 + E_2 = \sqrt{\frac{e * u}{s}} - e + \frac{O}{s}$.

\section{Minting Rules}

Based on needed reserves for worst-case scenario estimation, we can consider minting rules accordingly. Similarly to SigUSD~(a Djed instantiation), we can, for example, target for security in case of 
4x price drop, so to consider $R = \frac{p}{s} = 4$, and allow to mint \sct{}~while there are enough \bc{}~in reserves, so while there are not less than $E_w$ \bc{} in reserves. However, in this case most of time \sct{} would be non-mintable, and only during moments of \bc{} price going up significantly it will be possible to mint. 

Thus we leave worst-case scenario for UI, so dapps working with the \dx{} may show level of reserves, in comparison with worst-case scenario estimations. In this case, having on-chain data analysis, 
more precise estimations of reserve quality can be made~(by considering \sct{} locked in DeFi protocols, likely forgotten, etc).

We are proposing following minting rules.  

\begin{itemize}
  \item{Arbitrage mint: } if price reported by the oracle is higher than in the pool, i.e. $p > \frac{u}{e}$, we allow to print enough \sct{} to bring the price to $p$. That is, the bank allows to mint up to $\delta_u = \sqrt{p*e*u}-u$ \sct{} by accepting up to $\delta_e = \frac{\delta_u}{p}$ \bc{}s into reserves. 

  \em{To instantiate the rule, we can allow for arbitrage minting if the price $p$ is more than $\frac{u}{e}$ by at least 1\% for time period $T_{arb}$ (e.g. 1 hour), also, the bank is chargin 1\% fee 
  for the operation.} 

  \item{Free mint: } we allow to mint up to $\frac{u}{100}$ \sct{}s within time period $T_{free}$. 
  \em{To instantiate the rule, we propose to have bank fee of 1\%, and allow for free mint if $0.98 < r < 1.02$}.
\end{itemize}  


In addition to minting actions, which increase bank reserves only, we define following two actions which decrease them: 

\begin{itemize}
   \item{Intervention: } if reported by the oracle is lower than in the pool by significantly enough margin, i.e. $\frac{p*e}{u} < r$, where $r$ is some constant which is hard-wired into the protocol, then the bank is providing \bc{}s. to restore the price.
   \em{To instantiate the rule, we propose to allow the bank to intervene if $r <= 0.98$ for time period $T_{int}$}
   \item{Payout: } if bank has too much reserves, so $E > E_w$, we can allow for paying excess reserves out. There could be different ways to do this. E. g. extra reserves can be paid to holders of
   liquidity pool LP tokens.
\end{itemize}


We also state following rules for the liquidity pool (which, otherwise, acts as ordinary CP-AMM liquidity pool): 

\begin{itemize}
   \item{Stopping withdrawals: } if $r$ is below some threshold, withdrawals are stopped.  
   \em{To instantiate the rule, we propose to stop withdrawals immediately if $r <= 0.98$.}
   \item{Burn: } if the bank if empty, and $r$ is below some threshold, it is allowed to burn \sct{}s in the pool. 
   \em{To instantiate the rule, we propose to \sct{} burn if $r <= 0.95$ for time period $T_{burn}$. $T_{burn}$ must be quite big}
\end{itemize} 

\section{Notes on stability}
 
\knote{fill the section}

\section{Implementation}

\knote{put contracts here}

\section{Simulations}
We made simulations. \knote{finish}

\section{Extensions}


%% !TEX root = laws.tex

\newcommand{\avector}[2]{(#1_1,#1_2,\ldots,#1_{#2})}
\newcommand{\aDEFvector}[2][a]{(#1_1,#1_2,\ldots,#1_{#2})}

\subsection{Property-Based Testing}
In this section, we give a formal definition of a property followed by a discussion on property-based testing in contrast to conventional testing methodologies.

Within the scope of a data domain $\mathbb{D}$, a property can be seen as a collective abstract behavior which has to be followed by every valid member of the data domain. More precisely, a property is a predicate $P: \mathbb{D} \rightarrow \{true, false\}$ and it is desirable that it be \emph{valid}: 
\begin{center}
$\forall X \in \mathbb{D}, P(X) = true$
\end{center}
To illustrate, an example of a property $P$ over the domain of pairs of strings $\mathbb{S} \times \mathbb{S}$ is shown below:
\begin{center}
$P((s_1, s_2)) = \#(s_1::s_2) > \#s_1$
\end{center}
where $::$ denotes string concatenation and $\#s$ denotes the length of string $s$. This property is false for any $(s_1, \varepsilon)$, where $\varepsilon$ is the empty string. Therefore, it is not valid.

In contrast to conventional testing methods, where the behavior of a program is only tested on some pre-determined cases, property-based testing \cite{ron2001property} emphasizes defining properties and then testing their validity against randomly sampled data points. As property-based testing uses a small number of randomly sampled data points, it still provides only an approximate answer to the question of whether a property is satisfied on all data points. However, it may provide more confidence than conventional unit testing, because the randomly sampled data points may cover problematic cases that were not foreseen by the developers. There are various popular libraries available for property testing including QuickCheck for Haskell \cite{claessen2011quickcheck}, JUnit-QuickCheck for Java \cite{jung2015quickcheck}, theft for C, ScalaTest \cite{venners2009scalatest} and ScalaCheck \cite{nilsson2014scalacheck} for Scala.

Property-based testing is also advantageous when testing an application developed on top of a general framework, as is the case of blockchain systems developed on top of Scorex, because the framework may provide pre-implemented properties that the application should satisfy and the application developer just needs to implement application-specific generators of random data points.
\nocite{holzmann1995improvement}
\nocite{zaki2008formal}

%% !TEX root = laws.tex

\subsection{The Scorex Framework}

The idea of a modular design for a cryptocurrency was first proposed by Goodman in the Tezos position paper~\cite{tezosPosition}. The paper (in Section 2) proposes to split a cryptocurrency design into the three protocols: network, transaction and consensus. In many cases, however, these layers are tightly coupled and it is hard to describe them separately. For example, in a proof-of-stake cryptocurrency a balance sheet, which representation is heavily influenced by a transaction format, is used in a consensus protocol. 

Plenty of modular open-source frameworks were proposed for speeding up development of new blockchain systems, including: Sawtooth~\cite{sawtooth} and Fabric~\cite{fabric} by Hyperledger, Exonum~\cite{exonum} by Bitfury Group, and Scorex~\cite{scorex} by IOHK. We have chosen Scorex, because it has finer granularity. In particular, in order to support hybrid blockchains as well as more complicated linking structures than a chain~(such as Spectre\cite{spectre}), Scorex does not have the notion of blockchain as a core abstraction. Instead, it provides a more general abstract interface to a \textit{history} which contains \textit{persistent modifiers}\footnote{In a blockchain-based cryptocurrency, the blockchain can be seen as the history and every block can be seen as a persistent modifier.}. The history is a part of a \textit{node view}, which is a quadruple of $\langle$\textit{history}, \textit{minimal state}, \textit{vault}, \textit{memory pool}$\rangle$. The node view is updated whenever a persistent modifier or a transaction is processed. The minimal state is a data structure and a corresponding interface providing the ability to check the validity of an arbitrary persistent modifier for the current moment of time with the same result for all the nodes in the network having the same history. The minimal state is to be obtained deterministically from an initial pre-historical state and the history. The vault holds node-specific information, such as a user's wallet. The memory pool holds unconfirmed transactions being propagated across the network by nodes before their inclusion into the history. Such a design, described in details in Section~\ref{sec:scorex}, gives us the possibility to develop an abstract testing framework where it is possible to state contracts for the node view quadruple components without knowing details of their implementations.
%% !TEX root = laws.tex

\subsection{Our Contribution}

This paper reports on the design and implementation of a suite of abstract property tests which are implemented on top of the Scorex framework to ease checking whether a blockchain client satisfies the specified properties. A developer of a concrete blockchain system just needs to implement generators of random test inputs~(for example, random blocks and transactions for a Bitcoin-like system), and then the testing system will extensively check properties against multiple input objects. We have implemented 59 property tests, and integrated them into a prototype implementation of the TwinsCoin~\cite{cryptoeprint:2017:232} cryptocurrency.  

%% !TEX root = laws.tex
\raggedbottom
\subsection{Related Work}

Verification and testing of software systems \cite{myers2011art} is an integral part of a software development lifecycle. Immediately after the implementation of the software, and before its deployment, it has to be verified and tested extensively enough to ensure that all the functional requirements have been properly met. Over the course of time, both testing and verification methods have been becoming increasingly formal, sophisticated and automated. 

%Formal verification methods \cite{wang2004formal}\bruno{is this citation appropriate?} have been gaining popularity and acceptance \cite{LiquidHaskell,Stainless,Coq}. 
Formal verification usually involves constructing an abstract mathematical model (a.k.a. specification) of the system's desired behavior. From a logical perspective, the specification can be regarded as a collection of properties that ought to hold for the system, although often the specification is not described directly in logical form, but rather using various mathematical modeling frameworks, such as finite state machines \cite{chow1978testing}, Petri Nets, process algebra and timed automata \cite{clarke1996formal}. Once both the specification and the system are ready, the actual verification that the program satisfies the specification can be attempted. If successful, the verification proves that the properties of the specification are valid for the program. This is a starkly stronger result than what can be achieved through testing, where the properties are only checked on a few samples. However, full verification is hard to achieve automatically, and expensive to do manually or interactively.

Testing (either conventional or property-based) remains a less costly and hence more prevalent approach. Since a software program is developed at module or class level and is integrated with other modules or classes along the development cycle, testing is done at unit level, integration level and system level \cite{myers2011art}, before the software is deployed. End-to-end testing \cite{tsai2001end} is also performed, usually after system testing, %~(sometimes it is seen as a kind of system testing as well), 
to validate correct flow spanning different components of the software in real world use cases. Unit tests target individual modules, methods or classes and have a small coverage compared to integration tests which aim towards checking the behavior of modules when combined together. The two main approaches to unit testing are black box testing and white box testing. The former one focuses on designing test instances without looking inside the code or design, in other words, the black box testing focuses only on the extensional functionality of the unit under testing, while the white box testing approach is more inclined towards code coverage (i.e. ensuring that test instances execute as many different paths of the code as possible).

Although initially white box testing was considered only as a method for unit testing, recently it has emerged as a popular method for integration testing as well. Integration testing is usually done by one or a combination of the following approaches: 
\begin{enumerate}[\IEEEsetlabelwidth{Z}]
\item \textit{Big-Bang approach:} all the components are integrated together at once and then tested. This method works well for comparatively smaller systems, but is not well suited for larger systems. One obvious disadvantage is that the testing can only begin after all the individual components have been built.
\item \textit{Top-Down approach:} the modules at upper level are tested first and then we move down until we test the lowest level modules at the end. Since lower level modules might not be developed when the upper ones are being tested, stubs are used in place of such modules. The stubs try to simulate behaviour of the modules not yet implemented.
\item \textit{Bottom-Up approach:} in contrast to the top-down approach, here the lower level modules are tested first and then we iteratively move upwards in the hierarchy until we reach the highest level module. Now as we are testing lower level modules first, stubs are used to simulate the behaviour of not yet implemented higher level modules, in case any sibling interaction is required.
\item \textit{Sandwich approach:} this combines the Bottom-Up and Top-Down approaches.
\end{enumerate}

Going beyond conventional unit testing methods, which do not take any input parameters, parameterized unit tests~\cite{tillmann2010parameterized} are generalized tests that have an encapsulated collection of test methods whose invocation and behaviour is controlled by a set of input parameters giving more flexibility and automation to unit testing as a whole.

The final full scale testing that a software product undergoes is called the system testing, which includes tests like security test, compatibility test, exceptions handling, scalability tests, stress tests and performance tests.

Stress tests are particularly important for electronic payment systems, even conventional ones that are not based on cryptocurrencies. Visa, for instance, performed an annual stress test in 2013 to prepare their VisaNet system for the peak traffic of the upcoming holiday season. The test results showed that the system was able to handle 47,000 transactions per second, a 56\% improvement compared to the system's capacity in the previous year. %[https://www.visa.com/blogarchives/us/2013/10/10/stress-test-prepares-visanet-for-the-most-wonderful-time-of-the-year/index.html].
Within cryptocurrencies, the Bitcoin network experienced a spam campaign called \textit{"stress test"}~\cite{baqer2016stressing}, which caused the network's performance to degrade and essentially resulted in a denial-of-service attack~\footnote{a cyber-attack on a system where the attack makes the system's resources unavailable or degrades their quality to a point where it becomes difficult or sometimes impossible for honest users to avail the resource}. The intention behind this campaign was to expose vulnerabilities of the network, particularly when facing spam attacks. The maximum transaction verification rate of a network under spam can be improved through clustering of transactions to differentiate spam and genuine transactions~\cite{baqer2016stressing} or through $UTXO$-cleanup transactions, a new special type of transaction created by miners to combine spam transactions together, thereby reducing the $UTXO$ set size and the impact of the spam attack on the network.

% ignored text begins
% \ignore{ Similar to this, another vulnerability in the Bitcoin system caused MtGox Bitcoin exchange to close in February 2014 [?https://www.businessinsider.in/Bitcoin-Just-Completely-Crashed-As-Major-Exchange-Says-Withdrawals-Remain-Halted/articleshow/30165462.cms]. The exchange announced that close to 850,000 bitcoins were stolen by an attacker who exploited the vulnerability that causes bitcoin transactions to be malleable. Let us denote a bitcoin transaction as the tuple $T = (M, sig$) where $M$ is the message content of transaction and $sig$ is a valid signature on $M$. If a transaction is non-malleable, then it is not viable to construct another transaction $T' = (M, sig')$ such that $sig'$ is also a valid signature on $M$, without the knowledge of the secret key. Due to the fact that in bitcoin a transaction is identified by its unique $id = \mathbb{H}(M, sig)$, where $\mathbb{H}$ is a hash function, and not just $id = \mathbb{H}(M)$ means that $T$ and $T'$ as mentioned above will be treated as different transactions since they will have different $id$, despite the fact that their transaction content is exactly same. The above malleability is possible due to the fact that signature schemes used in bitcoin can be malleable. The way this attack was used to steal money, as claimed by the exchange, is the following:
% \begin{enumerate}[\IEEEsetlabelwidth{Z}]
% \item A user begins by depositing a certain amount $a$ into exchange's account.
% \item He then asks the exchange to transfer his money back to him.
% \item The exchange issues a transaction $T$ to transfer $a$ bitcoins to the user's account.
% \item The user constructs another transaction $T'$ by exploiting the malleability of transactions.
% \item Suppose that somehow $T'$ gets included in the blockchain instead of $T$.
% \item This ensures that the user gets $a$ bitcoins in his account. But after this, the user files a request for resending the money claiming that he didn't receive it.
% \item To respond to this request, the exchange checks that no transaction with $id = id_T$ ($=\mathbb{H}(T)$) is present and reissues another transaction sending $a$ to the user. This way the user was able to receive double that coins that he were to receive without the vulnerability.
% \end{enumerate}
% A very intuitive solution [?https://eprint.iacr.org/2013/837.pdf] to this problem is to change bitcoin such that transactions are identified only with $M$ and not the input scripts (signatures). This would mean that even if a signature is forged, the new transaction will hash to the same $id$ as the previous transaction and would eliminate this issue. There is also another solution [?%https://fc15.ifca.ai/preproceedings/bitcoin/paper_9.pdf
% ] where malleability of bitcoin transaction is dealt-with specific to bitcoin contracts. }
% % ignored text ends


%Some papers to use for references particular to property testing - http://www.wisdom.weizmann.ac.il/~oded/test.html.%
%Integration testing papers for references - https://www.researchgate.net/profile/Xiaoying_Bai/publication/221028427_End-To-End_Integration_Testing_Design/links/02e7e516cabf5c969d000000.pdf%
%Formal verification - http://www.cerc.utexas.edu/~jay/fv_surveys/zaki-AMS-survey-FULL.pdf%
%Formal verification - http://www.cerc.utexas.edu/~jay/fv_surveys/wang_fvsurvey_timed_systems_proc_ieee2004.pdf%
%Unit testing - https://link.springer.com/article/10.1007/s10664-006-5964-9%
%Integration testing - https://link.springer.com/chapter/10.1007/978-3-540-31862-0_18%
%http://citeseerx.ist.psu.edu/viewdoc/download?doi=10.1.1.93.7961&rep=rep1&type=pdf%
%http://ieeexplore.ieee.org/document/1702519/%
%https://dl.acm.org/citation.cfm?id=1767341%




%% !TEX root = laws.tex


\subsection{Structure of the Paper}

In Section~\ref{sec:scorex} we explain the architecture of the Scorex framework. We then describe our approach to property-based testing of blockchain system properties and present many examples of blockchain property tests in Section~\ref{sec:props}. Finally we state our conclusions in Section~\ref{sec:conclusion}.

%% !TEX root = laws.tex

\section{Scorex Architecture}
\label{sec:scorex}

Scorex is a framework for rapid implementation of a blockchain protocol client. A client is a node in a peer-to-peer network. The client has a local view of the network state. The goal of the whole peer-to-peer system~\footnote{concretely, its honest nodes. For simplicity, we omit a notion of adversarial behavior further.} is to synchronize on a part of local views which is a subject of a consensus algorithm. Scorex splits a client's local view into the following four parts: 

\begin{itemize}
\item{\em history} is an append-only log of {\em persistent modifiers}. A modifier is persistent in the sense that it has a unique identifier, and it is always possible to check if the modifier was ever appended to the history~(by presenting its identifier). There are no limitations on a modifier structure, besides the requirements to have a unique identifier and at least one parent~(referenced by its identifier). A persistent modifier may contain transactions, but this is optional. A transaction, unlike a persistent modifier, has no mandatory reference to its parents; also we consider that a transaction is not to be applied to the history and a minimal state~(described below). If a modifier is applicable to a history instance and so could be appended to it, we say that the modifier is {\em syntactically} valid. As an example, in a Bitcoin-like blockchain the history is about a chain of blocks. A block is syntactically valid if its header is well-formed according to the protocol rules, and current amount of work was spent on generating it. However, a syntactically valid block could contain invalid transaction, see a notion of semantic validity below. We note that there are alternative blockchain protocols with multiple kinds of blocks, microblocks, paired chains, and so on, that is why we have chosen abstract notions of a persistent modifier and a history, not the block and the blockchain.   

\item{\em minimal state} is a structure enough to check semantics of an arbitrary persistent modifier with a constraint that the procedure of checking has to be deterministic in nature. If a modifier is valid w.r.t minimal state, we call it a {\em semantically} valid modifier.
Thus, in addition to syntax of the blockchain, there is some stateful semantics, and minimal state takes care of it. That is, all nodes in the system do agree on some pre-historical state $S_0$, and then by applying the same sequence of persistent modifiers $m_1, \ldots, m_k$ in a deterministic way, all the nodes get the same state $S_k = apply(\ldots apply(apply(S_0, m_1), m_2), m_k)$ if all the $m_1, \ldots, m_k$ are {\em semantically} valid; otherwise a node gets an error on the first application of a semantically invalid persistent modifier. From this more abstract point of view, the goal of obtaining the state $S_k$ is to check whether a new modifier $m_{k+1}$ will be valid against it or not. Thus the minimal state has very few mandatory functions to implement, such as $apply(\cdot)$ and $rollback(\cdot)$ (the latter is needed for forks processing).

\item{\em vault} contains user-specific information. For a user running a node, the goal to run it is usually to get valuable user-specific information from processing the history. For that, the vault component is used which has the only functions to update itself by scanning a persistent modifier or a transaction and also to rollback to some previous state. A wallet is the perfect example of a vault implementation. 

\item{\em memory pool} is storing transactions to be packed into persistent modifiers.
\end{itemize}

The history and the minimal state are parts of local views to be synchronized across the network by using a distributed and decentralized consensus algorithm. Nodes run a consensus protocol to form a proper history, and the history should result in a valid minimal state when persistent modifiers from the history are applied to a publicly known prehistorical state.

The whole node view quadruple is to be updated atomically by applying either a persistent node view modifier or an unconfirmed transaction. Scorex provides guarantees of atomicity and consistency for the application while a developer of a concrete system needs to provide implementations for the abstract parts of the quadruple as well as a family of persistent modifiers.

A central component which holds the quadruple {\em <history, minimal state, vault, memory pool>} and processes its updates atomically is called a {\em node view holder}. The holder is processing all the received commands to update the quadruple in sequence, even if they are received from multiple threads. If the holder gets a transaction, it updates the vault and the memory pool with it. Otherwise, if the holder gets a persistent modifier, it first updates the history by appending the modifier to it. In a simplest case, if appending is successful~(so if the modifier is syntactically valid), the modifier is then applied to the minimal state. However, sometimes a fork happens, so the state is needed to be rolled back first, and then a new sequence of persistent modifiers is to be applied to it. 

As an example, we consider the cryptocurrency Twinscoin~\cite{cryptoeprint:2017:232}, which is based on a hybrid proof-of-work and proof-of-stake consensus protocol. Scorex has a full-fledged Twinscoin implementation as an example of its usage. There are two kinds of persistent modifiers in Twinscoin: a proof-of-work block and a proof-of-stake block. Thus the blockchain is hybrid: after a Proof-of-Work block it could be only a Proof-of-Stake block, and on top of it there could be only a Proof-of-Work block again. Thus a TwinsCoin-powered blockchain is actually two chains braided together. Only Proof-of-Stake block could contain transactions. Such complicated design makes Scorex a good framework to implement the TwinsCoin proposal. Unfortunately, TwinsCoin authors made only some particular tests. We got working tests for the Twinscoin client just by writing generators for transactions and persistent modifiers.

It could be the case that in a decentralized network two generators are issuing a block at the same time, or in the presence of a temporary network split different nodes are working on different suffixes starting with the same chain, or an adversary may generate blocks in private and then present them to the network. In short, a fork could happen. This is a normal situation once majority of block generators are honest~(see~\cite{Garay2015} for formal analysis of the Bitcoin proof-of-work protocol).    

Processing forks in a client could be a complicated issue, making testing of this functionality important. We proceed by describing the way in which forking is implemented in Scorex. When a persistent modifier is appended to a history instance, the history returns~(if the modifier is syntactically valid) {\em progress info} structure which contains a sequence of persistent modifiers to apply as well as a possible identifier of a modifier to perform rollback~(for the minimal state, vault, memory pool) to before the application of the sequence. By such a realization of the interfaces, Scorex allows history to be non-linear~(for example, it could be a block tree), but other components of the node view quadruple have sequential logic. For efficiency reasons, the minimal state is usually limited in maximal depth for a rollback, so the rollback could fail~(this situation is probably unresolvable in a satisfactory way without a human intervention). 

%% !TEX root = laws.tex

\section{Property-based testing of a blockchain client}
\label{sec:props}

In this section we report on our approach to generalized exhaustive testing of an abstract blockchain~(or blockchain-like) protocol implementation. For extensive testing, we test history, minimal state and memory pool components separately, and also do thorough checks for node view holder properties.
	
In total, we have implemented 59 property tests. They are using random object generators described in Section~\ref{sec:generators}. Most of the tests are relatively simple, others could check complex functionalities where several components are involved. We provide many examples in Section~\ref{sec:examples}.

\subsection{Generators}
\label{sec:generators}

We recall that (unlike unit tests, for example), property-based tests are checking not an output of a functionality under test against a concrete input, but rather a relation between input and output values for an arbitrary input value. Thus, in order to run a property-based test, an instance of an input value is needed. To be able to obtain it, a property-based test is supplied with a random input generator, which provides a random input domain element upon request. For our testing framework, a developer of a concrete protocol client needs to provide implementation for generators of the following types:

\begin{itemize}
	\item{a syntactically valid (respectively, invalid) modifier, which is valid (respectively, invalid) against given history instance}
	\item{a semantically valid (respectively, invalid) modifier, which is valid (respectively, invalid) against given minimal state instance. The modifier could be syntactically invalid}
	\item{a totally, so both semantically and syntactically, valid modifier. Respectively, a sequence of totally valid modifiers}
	\item{a transaction}
	\item{history instance, for which it should be possible to generate a syntactically valid modifier}
	\item{minimal state instance, for which it should be possible to generate a semantically valid modifier}
	\item{vault instance}
	\item{node view holder instance, for which it should be possible to generate a totally valid modifier}
\end{itemize}

As an example, for the TwinsCoin implementation we provide concrete implementations for all the generators mentioned above. To generate a syntactically valid modifier, we generate a Proof-of-Work block if a previous pair of {\em<Proof-of-Work block, Proof-of-Stake block>} is complete, otherwise we generate a new Proof-of-Stake block. We recall that in TwinsCoin transactions can be recorded only in Proof-of-Stake blocks. A minimal state in the TwinsCoin implementation, similarly to Bitcoin, is defined as a set of current unspent transaction outputs. In order to generate a semantically valid modifier, we generate a Proof-of-Stake block including transactions based on unspent transaction outputs. A totally valid modifier generator, based on given history as well as minimal state instances, produces either an empty Proof-of-Work~\footnote{unlike Bitcoin, Twinscoin does not have a notion of a coinbase transaction rewarding miner, instead, block generator's public key is included into the block directly.} or a semantically valid Proof-of-Stake block, depending on the last block in the history~(in order to generate the modifier which is also syntactically valid). 

Interestingly, we implicitly define some properties via generators. In particular, the existence of a generator for a totally valid modifier for any given correct history and valid minimal state instances assumes that it is always possible to make a progress in constructing a blockchain. To the best of our knowledge, the need of this property to be hold was first stated in the formal model of the Bitcoin protocol by Garay et. al.~\cite{Garay2015}~(see ``Input Validity'' definition in Section 3.1 of the paper~\cite{Garay2015}). 


\ignore{

\subsection{Forking}
\label{sec:forking}


We have implemented forking tests for the node view holder. The tests are checking that shorter sequence of totally valid persistent modifiers is not resulting in a fork, a longer sequence leads to a fork. A separate test is trying to apply a fork longer than the maximal rollback depth and checks whether the node view holder is emitting a needed alert.

\subsection{Simple properties}
\label{sec:simple-props}

We provide some examples of simple properties we are checking. If a persistent modifier has been appended to a history, it is always possible to get it at some later point of time, for any implementation of the history interface and any syntactically valid modifier. In opposite, if modifier is syntactically invalid, it should be not possible to get it from the history by request. Similarly, if a persistent modifier is semantically valid against a minimal state, the former could be applied to the latter. Even more, after application we can, roll the application back and successfully apply the persistent modifier again.  

For a {\em node view synchronizer} component, which acts as a proxy layer between the node view holder and networking protocol, we check, in particular, that if a transaction or a persistent modifier is coming in from a simulated ``peer'', the node view holder is getting it within a reasonable timeout. Similarly, if a transaction or a persistent modifier is coming from simulated local side, the synchronizer should send it to a network layer within a timeout.

}
%% !TEX root = laws.tex

\subsection{Examples of Properties Tests}
\label{sec:examples}

To explain our approach to the testing of a client in details, in this section we provide some examples of property tests which are valid for most blockchain-based systems. We have grouped the tests based on their similarity.\\

\begin{enumerate}[\IEEEsetlabelwidth{Z}]

\item \textit{Memory Pool Tests}.\\
Memory Pool (or just mempool in the Bitcoin jargon) is used to store unconfirmed transactions which are to be included into persistent modifiers. The following tests are used to check some general properties of a memory pool which every blockchain client should pass.

\begin{itemize}[\IEEEsetlabelwidth{Z}]

\item \textit{A memory pool should be able to store enough transactions:} in TwinsCoin implementation, we are testing that the memory pool which is empty before the test should be able to store a number of transactions up to a maximum specified in settings.\\

\item \textit{Filtering of valid and invalid transactions from a memory pool should be fast:} we got an impression from running the Twinscoin client that memory pool probably spends too much time on filtering out a transaction. To be certain about that we have implemented a test which is checking that an implementation of memory pool is able to filter out a transaction reasonably fast. As processing time is platform-dependent, the test during its instantiation is measuring time to calculate 500,000 blocks of SHA-256 hash. Time to filter out the transaction should be no more than that. We found that the Twinscoin implementation was really inefficient about filtering.\\

\item \textit{A transaction successfully added to memory pool should be available by a transaction identifier:} the purpose of this test is to ensure that once a transaction is added to the memory pool, it indeed is available by a transaction identifier. The test simply adds the transaction to the memory pool and then query the transaction by its identifier. The initial transaction is the only correct result of the compound operation. \\
\end{itemize}
\item \textit{History Tests}.\\
A history is an abstract data structure which records all the persistent modifiers ever appended to it. We recall that the blockchain structure in the Bitcoin protocol is the example of a history implementation. Since history is an integral part of a node view, it is important to check if an implementation of history acts correctly. A consensus protocol aims at establishing common history for all the nodes on the network.

A persistent modifier is the main building block of a history and is used to update the history and a minimal state. As soon as a valid modifier got appended to history, the whole node's local view is being changed in the sense that the history is updated, possibly along with the minimal state.

We have many tests to test history, some examples are provided below.

\begin{itemize}[\IEEEsetlabelwidth{Z}]
\item \textit{A syntactically valid persistent modifier should be applicable to a history instance and available by its identifier after that:} by definition, a syntactically valid persistent modifier should be applicable to a history instance, and then, once applied to the history, it should be available by its unique identifier. The importance of this test comes from the fact that it is of utmost importance for the client implementation to tell the difference between the modifiers that have been appended to the history from those that have not been added. For this purpose, the unique identifier of the modifier can be used to query the history to know whether the modifier has been added to the history of not.\\

\item \textit{A syntactically invalid modifier should not be able to be added to history:} a syntactically invalid modifier should not be applicable to a history instance. The test first checks whether application of the modifier returns an indication of an error. Then the test checks that the modifier should not be available by its identifier.\\

\item \textit{Modifier not ever appended to a history instance should not be available by request:} when the persistent modifier which was never appended to the history, is queried from the history, it should always return empty result which shows that the invalid modifier has not been added to the history.\\

\item \textit{After application of a syntactically invalid modifier to a history instance, it should not be available in history by its identifier:} a syntactically invalid modifier is one which is inconsistent with the present view of history. Only syntactically valid modifier is eligible to be applied to history. To check how the history is filtering out invalid modifiers, we propose this test. We generate a random invalid modifier and attempt to add it to history. Since it is invalid, history should not add it and hence, it should not be available by identifier when queried from history. Some examples of generated invalid modifiers are ones with false nonces which do not satisfy the puzzle and ones with non-valid signatures.\\

\item \textit{Once a syntactically valid modifier is appended to history, the history should contain it:} this test ensures that if valid modifiers are correctly appended to the history, then they should be then available by their respectable identifiers. Also, the test is checking that the history is indicating success during the application.\\

\item \textit{History correctly reports semantic validity of an identifier:} a history instance should be able to indicate semantic validity of a persistent modifier. If the modifier is not appended to the history yet, the history should return on request that semantic validity of the modifier is not known. The same result should be returned once the modifier is appended, but semantic validity status is not provided by the node view holder~(after applying the modifier to the minimal state). Once semantic validity status is provided, the history should return it by request. The test checks all the options, simulating node view holder with a stub.\\   

\end{itemize}

\item \textit{Minimal State Tests.}\\
Tests for the minimal state component are checking application of a semantically valid~(respectively, invalid) persistent modifier, and also rollback functionality. In case of a better version of history~(a fork) found, a rollback has to be performed which essentially rolls the system back to a common point~(from which forks are started). Known examples of rollbacks performed in the Bitcoin network are recovery from the SPV mining issue~\cite{spvMining}, and also recovery from the arithmetic overflow bug~\cite{overflow}.

\begin{itemize}[\IEEEsetlabelwidth{Z}]

\item \textit{Application and rollback should lead to the same minimal state:} in this test, a semantically valid persistent modifier $m$ is generated and applied to a current state $S$. Due to this application of the modifier, a new minimal state $S'$ is to be obtained from $S$. After the modifier is applied successfully to the history, a rollback is performed to take the system back to state $S$ from the current state $S'$. The test now checks that the state to which the system comes after the rollback is indeed the state $S$ by checking an identifier of the new state after rollback is the same as the identifier of the original state.\\

We now use this test to explain how we generate a semantically valid modifier for the Twinscoin client. Before proceeding, we define the structure of a transaction $t$. A simple transaction is usually represented as the map $T : UTXO \to UTXO$, where $UTXO$ is the set of all the unspent transaction outputs or \textit{boxes}. A box can be considered as a tuple ($pubkey, amount$) where $pubkey$ is the public key of the account of the node to which this box belongs to and $amount$ refers to the monetary (in case of cryptocurrencies) amount which this box holds in the name of the $pubkey$ in the first half of the tuple. A transaction uses some ($\geq 0$, 0 in the case of the rewarding transaction which is present at the end of each block and which rewards the miner) unspent boxes and generates new boxes with a constraint that sum of the amount of all the boxes used in the transaction is equal to the sum of amount of the boxes output by the transaction except the rewarding transaction for which this constraint doesn't apply. Once the new boxes have been added to the UTXO set, the old boxes which were input to the transaction are removed from the UTXO set to prevent double spending. This addition and removal of boxes from $UTXO$ set has to be done atomically in order to avoid inconsistencies in the system. Suppose a node $A$ wants to send $x$ amount to a node $B$, then the transaction for this purpose will use some boxes with $A$s public key on them which sum up to an amount $y \geq x$ and output the boxes $b_1$ and $b_2$ such that $b_1 = (B, x)$ which has $B$s public key and an amount $x$ whereas $b_2 = (A, y-x)$ will have $A$s public key and an amount of $y-x$, if $y>x$. Now $B$ has received a new box $b_1$ which belongs to him and this box now sits inside the set $UTXO$ until the point when $B$ uses this box an one of the inputs to a future transaction. Readers should note that for more readability we will represent a transaction $T$ by a tuple ([$in_1, in_2, ...$], [$o_1, o_2, ...$]) where $in_i$ denotes input boxes to the transaction and $o_i$ denotes the output boxes of the transaction.\\
Along with checking that the $id$s are same, it also checks that the components of the new state are also rolled back and not just the $id$ number got rolled back. For this, we generate the modifier $m$ in the following way:

\begin{itemize}
\item Generate a pair of transactions ($t_1, t_2$) where $t_1 = ([b_{1}], [b_{2}])$ and $t_2 = ([b_{2}], [b_{3}])$. This notation means that the first transaction $t_1$ uses a box $b_1$ as its input and then outputs a box $b_2$ which is then used by the second transaction as its input. In the above setting, we select the first input box $b_1$ randomly from the set $UTXO$ and finally output the box $b_3$ from $t_2$ which also just generates a random box ($b_3$). The generated transaction pair has to be valid in the sense that it should only use valid unspent boxes from $UTXO$ and satisfy the constraint that the sum of amounts of all the input boxes should be equal to the sum of amounts of the output boxes. The main caveat here is that the second transaction of the pair should use the output box of the first transaction of the pair.
\item Now we generate a pair of modifiers ($m_1, m_2$) and include both of these transactions from the pair above in the respective modifiers.
\end{itemize}

Once the custom modifiers are generated, $m_1$ (first half of the pair) is appended to the history and the system moves from state $s_1$ to the state $s_2$. As mentioned before, the transaction $t_1$ from the pair uses a random box $b_1$ from the $UTXO$ of state $s_1$ and when the system moves to the state $s_2$, the $UTXO$ gets added with the box $b_2$ and $b_1$ is removed from the set. Once the state change happens, we append $m_2$ (second half of the modifier pair) to history progressing the system to state $s_3$. Since $m_2$ contains the transaction $t_2$ which takes as input $b_2$, when the system moves to $s_3$, $b_2$ is removed from $UTXO$ and $b_3$ is added.\\

Now it becomes clear why we generate pairs of transactions and modifiers in the way defined above. Finally, we perform a rollback from state $s_3$ to $s_2$ which should mean that once the rollback is successful, the box $b_2$ should come back to the set $UTXO$ and should be available by $id$ whereas the box $b_3$ should now not be present inside the $UTXO$ set anymore. Both of these checks tell us that the rollback was performed correctly and the system indeed came back to the previous state with all its components. The reason that we generated the pairs of transactions above is because it helps us in easily checking by $id$ if $b_2$ has returned to the $UTXO$ set since we generated $b_2$ ourselves and know its $id$ already. This makes testing easy and transparent. \\

\item \textit{Application of a valid modifier again after a rollback should be successful:} as the previous test aimed at checking that the components of a state are recovered after a rollback happens, it would be quite wrong to think that it should be the only test that is necessary to check if the rollback system performs as expected. It is also equally important that after rollback the system performs normally, as it would perform if the rollback would have never happened. To check this property to certain degree, we propose this test. It checks that after the rollback has happened the system becomes stable again and any new valid modifier which is now added to the history is actually recorded and hence should be available if queried from history. This test ensures that after recovering from a rollback the system performs normally and can resume its functioning without any issues. It hence ensures that a continuity is maintained after a rollback.\\

\item \textit{Double application of a semantically valid modifier should not be possible:} this test checks that a semantically valid persistent modifier should not be added more than once. For example, if in Bitcoin a block could be successfully applied twice to the validation state~(UTXO set), all the transaction inside the modifier will be double spent. We argue that an implementation of a blockchain system should prevent addition of a semantically valid persistent modifier twice in general. In this test, we generate a semantically valid modifier, then append it to a generated minimal state once, and on the second application of the modifier again to the minimal state an error should be returned.\\

\end{itemize}

\item \textit{Node View Holder Tests.}\\
As was mentioned in Section~\ref{sec:scorex}, the node view holder is the central component of a blockchain client which is responsible for atomically updating the quadruple \textit{<history, minimal state, vault, memory pool>}. The update could be triggered by whether a persistent modifier of a transaction coming in. Below we provide some implemented tests for the node view holder. % We recall that for a successful update of the quadruple, a totally valid persistent modifier is needed, which is both syntactically and semantically valid.


\begin{itemize}[\IEEEsetlabelwidth{Z}]

\item \textit{A totally valid persistent modifier should successfully update the minimal state and the history:} we recall that a totally valid modifier is a persistent modifier which is valid for both the history and the minimal state, so it is applicable to both of them. In this test we are sending a random totally valid persistent modifier to the node view holder component and then we are checking that history contains it and the version of the minimal state is equal to the modifier's identifier.\\
\end{itemize}

\item \textit{Forking tests.}\\
At the moment we have two tests for forking. In both of them we apply random number of totally valid modifiers in the first place and remember last block which we call the common block. One test then is applying two totally valid modifiers which have the common block as parent. The test checks that after the application the history contains whether one of these modifiers, and version of the minimal state is equals to identifier of whether one of the two modifiers. The logic behind such a check is that we do not know whether an implementation of a node view holder will make switching from one prefix~(of length one) to another~(of the same length), but anyway the general property should hold. Another test first generates a sequence of totally valid persistent modifiers of length 2, applies it to the common block, then the test generates a sequence of totally valid persistent modifiers of length 4 starting from the common block, and applies the longer sequence. The test checks that switching takes place, so minimal state version equals to the identifier of the last block in the longer sequence, and the history contains the identifier in the current modifiers which do not have ancestors~(for a blockchain, there is one such a modifier, for a block tree, there could be more than one modifiers returned). With the help of the forking tests we have found few errors in the Twinscoin implementation. 


\end{enumerate}



%- Valid box should be successfully applied to state, it's available by identifier after that.
%- State should be able to generate changes from valid block and apply them.
%- Wallet should contain secrets for all it's public propositions.
%-
%- Transactions once added to a block should be removed from the local copy of memory pool.
%- Minimal state should be able to add and remove boxes based on received transaction's validity.
%-
%- 
%- %- BlockchainSanity test that combines all this test.

%% !TEX root = laws.tex

\section{Conclusion}
\label{sec:conclusion}

In this paper we propose to improve quality of blockchain protocol implementations via exhaustive property-based testing. For generic abstract modular Scorex framework, we have implemented a suite of property-based tests. The suite consists of 59 tests checking different properties of a blockchain system. To run the suite against a concrete blockchain protocol client, developers of the client need to provide generators for random objects used by the protocol. The suite is checking properties against the implementation by using random samples. We used Twinscoin implementation provided with Scorex as an example of a concrete blockchain using our testing kit. In the paper we provide many examples of the tests.  

\section*{Acknowledgments}

Authors would like to thank.

\bibliographystyle{IEEEtran}
\bibliography{sources, ref}

%Appendices

%\newpage
%\appendix
%% !TEX root = laws.tex

\section{Tests Implemented}

\knote{check the lists: }

Implemented test scenarios
- Valid persistent modifier should be successfully applied to history and available by id after that.
- Valid box should be successfully applied to state, it's available by id after that.
- State should be able to generate changes from valid block and apply them.
- Wallet should contain secrets for all it's public propositions.
- State changes application and rollback should lead to the same state and the component changes should also be rolled back.
- Transactions successfully added to memory pool should be available by id.
- Transactions once added to a block should be removed from the local copy of mempool.
- Mempool should be able to store a lot of transactions and filtering of valid and invalid transactions should be fast.
- Minimal state should be able to add and remove boxes based on received transaction's validity.
- Modifier (to change state) application should lead to new minimal state whose elements' intersection with previous ones is not complete (at least some new boxes are introduced and some previous ones removed).
- Application of the same modifier twice should be unsuccessful.
- Application of invalid modifier (inconsistent with the previous ones) should be unsuccessful.
- Application of a valid modifier after rollback should be successful.
- Invalid modifiers should not be able to be added to history.
- Once an invalid modifier is appended to history, then history should not contain it and neither should it be available in history by it's id.
- History should contain valid modifier and report if a modifier if semantically valid after successfully appending it to history.
- BlockchainSanity test that combines all this test.

Coming test scenarios:
- Block application and rollback leads to the same history (rollback is not defined for history yet)
- NodeView apply block to both state and history or don't apply to any of them
- It is not possible to apply transaction to a state twice
- Tests for invalid transactions/blocks/etc

\knote{generators list}


\end{document}
