\documentclass{article}   % list options between brackets

\usepackage{color}
\usepackage{graphicx}
%% The amssymb package provides various useful mathematical symbols
\usepackage{amssymb}
%% The amsthm package provides extended theorem environments
%\usepackage{amsthm}
\usepackage{amsmath}

\usepackage{listings}

\usepackage{hyperref}

\usepackage{systeme}

\def\shownotes{1}
\def\notesinmargins{0}

\ifnum\shownotes=1
\ifnum\notesinmargins=1
\newcommand{\authnote}[2]{\marginpar{\parbox{\marginparwidth}{\tiny %
  \textsf{#1 {\textcolor{blue}{notes: #2}}}}}%
  \textcolor{blue}{\textbf{\dag}}}
\else
\newcommand{\authnote}[2]{
  \textsf{#1 \textcolor{blue}{: #2}}}
\fi
\else
\newcommand{\authnote}[2]{}
\fi

\newcommand{\knote}[1]{{\authnote{\textcolor{green}{Alex notes}}{#1}}}
\newcommand{\mnote}[1]{{\authnote{\textcolor{red}{Amitabh notes}}{#1}}}

\usepackage[dvipsnames]{xcolor}
\usepackage[colorinlistoftodos,prependcaption,textsize=tiny]{todonotes}


% type user-defined commands here
\usepackage[T1]{fontenc}

\usepackage{flushend}

%\usepackage{xcolor}

\newcommand{\bc}{ERG}
\newcommand{\stc}{stablecoin}
\newcommand{\sct}{stablecoin}
\newcommand{\dx}{Dexy}

\newcommand{\ma}{\mathcal{A}}
\newcommand{\mb}{\mathcal{B}}
\newcommand{\he}{\hat{e}}
\newcommand{\sr}{\stackrel}
\newcommand{\ra}{\rightarrow}
\newcommand{\la}{\leftarrow}
\newcommand{\state}{state}

\newcommand{\ignore}[1]{} 
\newcommand{\full}[1]{}
\newcommand{\notfull}[1]{#1}
\newcommand{\rand}{\stackrel{R}{\leftarrow}}
\newcommand{\mypar}[1]{\smallskip\noindent\textbf{#1.}}

\begin{document}

\title{Dexy - A Stablecoin With Algorithmic Central Bank}
\author{Alexander Chepurnoy, Amitabh Saxena}


\maketitle

\begin{abstract}
In this paper, we consider a new stablecoin protocol design called \dx. The protocol has two mandatory components, reference market~(in form of liquidity pool where anyone can 
trade base currency against stable coins), and an algorithmic central bank, which is responsible for new stablecoins issuance, and also for stabilizing stablecoin price in the reference market. 
\end{abstract}

% % !TEX root = laws.tex

\section{Introduction}

Algorithmic stablecoins is a natural extension of cryptocurrencies, trying to 
solve problems with volatility of their prices by pegging stablecoin price to an
asset which price believed to be more or less stable with time (e.g. 1 gram of gold).  


Stable pricing could be useful for:
\begin{enumerate}
\item securing fundraising;
\item doing business with predictable results. For example, a hosting provider knows 
\item shorting: .
\item lending and other decentralized finance applications. 
\end{enumerate}

Algorithmic stablecoins are different from centralized stablecoins, such as USDT and USDC, which are 
convertible into underlying (pegged asset) via a trusted party. In case of an algorithmic stablecoin, its 
pegging is done via rebasement of total supply, or via imitating the trusted party, which holds. 


\section{Introduction}

Algorithmic stablecoins is a natural extension of cryptocurrencies, trying to 
solve problems with volatility of their prices by pegging stablecoin price to an
asset which price is considered to be more or less stable with time (e.g. gold).  

Having an asset with a stable value could be useful for:
\begin{itemize}
\item securing fundraising; a project can be sure that funds collected during fundraising will have stable value in the mid- and long-term.
\item doing business with predictable results. For example, a shop can be sure that funds collected from sales will be about the same when the shop is ordering goods from warehouses~(otherwise, the shop may go bankrupt if margin is not that big). 
\item shorting: when cryptocurrency prices are high, it is desirable for investors to rebalance their portfolio by increasing exposure to fiat currencies (or traditional
 commodities). However, as fiat currencies and centralized exchanges impose significant risks, it would be better to buy fiat and commodity substitues in form of stablecoins on decentralized exchanges.
\item lending and other decentralized finance applications. Stability of collateral value is critical for many applications.
\end{itemize}

Algorithmic stablecoins are different from centralized stablecoins, such as USDT and USDC, for which there is a trusted party doing conversions to a pegged asset. In case of an algorithmic stablecoin, the 
pegging is done via rebasement of total supply~(as in Ampleforth), or via imitating the trusted party, which holds reserves and doing market interventions when it is needed for getting exchange rate back to the peg. Imititating the trusted party is usually done by allowing anyone on the blockchain creating over-collateralized financial instruments, such as collateralized debt positions~(as 
in DAI) or zero-coupon bonds~(as in the Yield protocol).

\knote{add links to the paragraph above}

In this work we present \dx{}, a stablecoin protocol where the bank presented explicitly as a contract with few predefined rules. The bank is trying to stabilize stablecoin value on the markets, using a liquidity pool as a reference market, by providing stablecoin liquidity, when stablecoin is over the peg, or injecting base currency from its reserves, when the stablecoin is under the peg. In extreme case, when bank reserves are depleted and stablecoin is still under the peg, its value is restored by burning stablecoin in the liquidity pool. In some aspects \dx{} could resemble Fei or Gyroscope. \knote{make comparison subsection}  


\section{\dx{} Design in General}

Unlike popular algorithmic stablecoins based on two tokens, \dx{} is based on one token but two protocols. In the first place, 
there is reference market~(done as on-chain protocol), where trading of \dx{} vs the base currency (\bc{}) happens. In the second place, if market price is way too different from target price (as reported by a trusted oracle), there is an algorithmic central bank which is doing interventions in order to readjust the market price~(so make it closer to the oracle's one). The central bank can also mint new \dx{} tokens, selling them for \bc{}. The bank is using reserves in \bc{} it is buying for interventions then. 

As a simple solution for the {\em reference market}, we are using constant-product Automated Market Maker (CP-AMM) liquidity pool, similar to ErgoDEX and UniSwap. The pool has \bc{} on one side and \dx{} on another. For CP-AMM pool, multiplication of \bc{} and \dx{} amounts before and after a trade should be preserved, so $e * u = e' * u'$, where $e$ and $u$ are amounts of \bc{} and \dx{} in the pool before the trade, and $e'$ and $u'$ are amounts of \bc{} and \dx{} in the pool after the trade, correspondingly. As for any CP-AMM pool, it is possible to put liquidity into the pool, and remove liquidity from it, however, there are some limitations here for \dx{} reference market we're going to uncover further. 

The bank has two basic operations. It can mint new \sct{} tokens per request, using trusted oracle's price, by accepting \bc{} in its reserves. It also can intervene into markets by providing \bc{} from reserves when needed~(namely, when price in the pool $\frac{u}{e}$ is significantly different from price on external markets $p$ which reported by oracle).

Now we are going to consider how to put restrictions and design rules for the system to ensure stable pricing for \sct{} tokens. 

\section{Notation}

We start with introducing notation: 
\begin{itemize}
  \item{} $T$ - period before intervention starts. After one intervention the bank can start another one after $T$. 
  \item{} $p$ - price reported by the oracle at the moment (for example, 20 USD per ERG)
  \item{} $s$ - price which the bank should stand in case of price crash. For example, we can assume that $s = \frac{p}{4}$ (so if p is 20 USD per ERG, then $s$ is 5 USD per ERG, means the bank needs to have enough reserves to save the markets when the price is suddenly crashing from 20 to 5 USD per ERG)
  \item{} $R$ - ratio between $p$ and $s$, $R = \frac{p}{s}$
  \item{} $r$ - ratio between $p$, and price in the pool, which is $\frac{u}{e}$, thus $r = \frac{p}{\frac{u}{e}} = \frac{p*e}{u}$
  \item{} $e$ - amount of \bc{} in the liquidity pool 
  \item{} $u$ - amount of \stc{} in the liquidity pool
  \item{} $O$ - amount of \stc{} outside the liquidity pool. The distribution in $O$ is not known for the \dx{} protocol, but the bank can easily store how many \sct{} tokens were minted and then get $O$ by deducting $u$ from it.
  \item{} $E$ - amount of \bc{} in the bank. 
\end{itemize}  

\section{Worst Scenario and Bank Reserves}

The bank is doing interventions when the situation is far from normal in the markets, and enough time passed for markets to stabilize themselves with no interventions. In our case, the bank is doing interventions based on stablecoin price in the liquidity pool in comparison with oracle provided price. The bank's intervention then is about injecting its \bc{} reserves into the pool.  

First of all, let's assume that the oracle price crashed from $p$ to $s$ sharply and stands there, and before the crash there were $e$ of \bc{} and $u$ of \sct{} in the liquidity pool, with pool's price being $p$. The worst case is when no liqudity put into the pool during the period $T$. With large enough $T$ and large enough $R$ this assumption is not very realistic probably: some traders will buy \bc{}s with their \sct{}s anyway, price is failing with swings where traders will mint \sct{}s by while increasing \bc{} bank reserves, etc. However, it would be reasonable to consider worst-case scenario, then in the real world \dx{} will be even more durable than in theory. 

In this case, the bank must intervene after $T$ units of time, as the price differs significantly, and restore the price in the pool, so set it to $s$. We denote amounts of \bc{} and \sct{} in the pool after the intervention as $e'$ and $u'$, respectively. Then:

\begin{itemize}
  \item{} $e * u = e' * u'$
  \item{} as the bank injects $E_1$ ergs into the pool, $e' = e + E_1$
  \item{} $\frac{u'}{e'} = s$, thus $u' = s * (e + E_1)$ 
  \item{} from above, $E_1 = \sqrt{\frac{e * u}{s}} - e$
\end{itemize}

So by injecting $E_1$ \bc{}, the bank recovers the price. However, this is not enough, as now there are $O$ \sct{} units which can be injected into the pool from outside. 
Again, in the real life it is not realistic to assume that all the $O$ \sct{} would be injected, as some of them are simply lost, some would be kept to buy cheap ERG at the bottom~(we remind that \sct{} often used as a bet for \bc{} price decline), etc. However, we need to assume worst-case scenario again. We also unrealistically assume that all the $O$ tokens are being sold in very small batches not significantly affecting price in the pool, and after each batch seller of a new batch is waiting for a bank intervention to happen (so for $T$ units of time), and sells only after the intervention. In this case all the $O$ tokens are being sold at price close to $s$, so the bank should have about $E_2 = \frac{O}{s}$ \bc{}s in reserve to buy the tokens back.

Summing up $E_1$ and $E_2$, we got \bc{} reserves the bank should have to be ready for worst-case scenario: $E_w = E_1 + E_2 = \sqrt{\frac{e * u}{s}} - e + \frac{O}{s}$.

It is simple to see why this scenario is worst-case for the bank. In this scenario, the bank~(and only the bank) is buying all the $O$ of external $\sct{}$ at the worst possible price $s$, and get to this price by burning its own reserves only.  

\section{Bank and Pool Rules}

Based on needed reserves for worst-case scenario estimation, we can consider minting rules accordingly. Similarly to SigUSD~(a Djed instantiation), we can, for example, target for security in case of 
4x price drop, so to consider $R = \frac{p}{s} = 4$, and allow to mint \sct{}~while there are enough, so not less than $E_w$, \bc{}~in reserves. However, in this case most of time \sct{} would be non-mintable, and only during moments of \bc{} price going up significantly it will be possible to mint~(there are simulations showing that which can be found in the \dx{} repository). As worst-case scenario is based on unrealistic assumptions, unlikely a realistic protocol can be built on top of it.  

Thus we leave worst-case scenario for UI, so dapps working with the \dx{} may show e.g. collateralization for the worst-case scenario. Having on-chain data analysis, 
more precise estimations of reserve quality can be made~(by considering \sct{}s locked in DeFi protocols, likely forgotten, etc).

We allow for cautios minting. That is, we whether allow minting when oracle price is above pool's price~(thus providing liquidity for arbitrage), or we allow to mint a little bit (per some time period) when liquidity pool is in good shape. In details, we have two following minting operations, with minting price being the oracle's price $p$:  

\begin{itemize}
  \item{Arbitrage mint: } if price reported by the oracle is higher than in the pool, i.e. $p > \frac{u}{e}$, we allow to print enough \sct{} tokens to bring the price to $p$. That is, the bank allows to mint up to $\delta_u = \sqrt{p*e*u}-u$ \sct{} by accepting up to $\delta_e = \frac{\delta_u}{p}$ \bc{}s into reserves. 

  \em{To instantiate the rule, we can allow for arbitrage minting if the price $p$ is more than $\frac{u}{e}$ by at least $1\%$ for time period $T_{arb}$ (e.g. 1 hour), also, the bank is charging $0.5\%$ fee for the operation. After arbitrage mint it is not possible to do another one within 30 minutes (to prevent aggressive liquidity minting via chained transctions etc).} 

  \item{Free mint: } we allow to mint up to $\frac{u}{100}$ \sct{}s within time period $T_{free}$. 
  \em{To instantiate the rule, we propose to have bank fee of 1\%, and allow for free mint if $0.98 < r < 1.02$. Minting fee in this case is $0.5\%$. We propose to set $T_free$ to $1$ day, then bank reserves can grow by $1\%$ of LP volume per day when the price is flat.}. 
\end{itemize}  

In addition to minting actions, which increase bank reserves, we define following two actions which decrease them: 

\begin{itemize}
   \item{Intervention: } if price reported by the oracle is lower than one in the pool by significantly enough margin, i.e. $\frac{p*e}{u} < r$, where $r$ is some constant which is hard-wired into the protocol, then the bank is providing \bc{}s. to restore the price.
   \em{To instantiate the rule, we propose to allow the bank to intervene if $r <= 0.98$ for time period $T_{int} = 1 {\ day}$.
       During intervention tracker is reset, so another intervention will be after $T_{int} = 1 {\ day}$ at least. }
   \item{Payout: } if bank has too much reserves, so $E > E_w$, we can allow for paying excess reserves out. There could be different ways to do this. For example, extra reserves can be paid to holders of liquidity pool's LP tokens via staking (then LP participant has potential source of additional revenue).
\end{itemize}


We also state following rules for the liquidity pool (which, otherwise, acts as ordinary CP-AMM liquidity pool): 

\begin{itemize}
   \item{Stopping withdrawals: } if $r$ is below some threshold, withdrawals are stopped, so only trades are possible.  
   \em{To instantiate the rule, we propose to stop withdrawals immediately if $r <= 0.98$.}

   \item{Second stabilization mechanism: } what if the bank is out of reserves, but \sct{} is still below the peg? In this case we restore price in the pool by removing liquidity, and there are two possible options here:

   \begin{enumerate}
   \item{Burn: } if the bank if empty, and $r$ is below some threshold, it is allowed to burn \sct{}s in the pool. 
   \em{To instantiate the rule, we propose to burn \sct{} if $r <= 0.95$ for time period $T_{burn}$. $T_{burn}$ must be quite big, e.g. $1 {\ week}$. We burn enough to return to the state of stopped withdrawals, so to $r = 0.98$. After burning the tracker is reset, so another burn will be done sfter $T_{burn}$ at least.}

   \item{Extract for the future: } if the bank is empty and $r$ is below some threshold, it is allowed to extract \sct{}s from the pool and lock by a contract which is releasing \sct{}s in the future when \sct{} price is above the peg.
   \em{To instantiate the rule, we propose to extract \sct{} if $r <= 0.95$ for time period $T_{burn}$ (so the same as in burn). 
   To prioritize extracted funds over arbitrage mint, we do not have delay in releasing contract. We burn enough to return to the state of stopped withdrawals, so to $r = 0.98$. After extraction the tracker is reset, so another burn will be done sfter $T_{burn}$ at least.}
   \end{enumerate}
\end{itemize} 

\section{Stability}

With second stabilization mechanism~(burning or extraction) in place, the price in the reference market will be eventually stabilized. However, for liquidity holders burning is painful, extraction not so but still not desirable, thus \dx{} protocol is trying to avoid it~(unlike other protocols, such as Gyroscope or Fei, where redemption rate fails below $1$ immediately as collateralization falls under $100\%$), by giving markets time to self-stabilize and then doing interventions. This could mean slower stabilization, in comparison with other protocols.

\dx{} is also cautios about minting new \sct{}s, which could mean slow growth of number of tokens issued. This could be inconvient, especially for big players, but the protocol is focused on stability in the first place. 

Please note, that liquidity pool is disincentivizing massive bank runs due to its constant-product nature. Massive bank runs are simpler for bank to resolve, in comparison with worst-case scenario. 
 
\knote{finish the section}

\section{Implementation}

\subsection{Bank Contract}
\knote{put contracts here}

\section{Simulations}
We made simulations, you can find them in \dx{} repository. \knote{finish}

\section*{Acknowledgments}

Authors would like to thank Ile for his inspiring forum posts.

\bibliographystyle{IEEEtran}
\bibliography{sources, ref}

%Appendices

%\newpage
%\appendix
%% !TEX root = laws.tex

\section{Tests Implemented}

\knote{check the lists: }

Implemented test scenarios
- Valid persistent modifier should be successfully applied to history and available by id after that.
- Valid box should be successfully applied to state, it's available by id after that.
- State should be able to generate changes from valid block and apply them.
- Wallet should contain secrets for all it's public propositions.
- State changes application and rollback should lead to the same state and the component changes should also be rolled back.
- Transactions successfully added to memory pool should be available by id.
- Transactions once added to a block should be removed from the local copy of mempool.
- Mempool should be able to store a lot of transactions and filtering of valid and invalid transactions should be fast.
- Minimal state should be able to add and remove boxes based on received transaction's validity.
- Modifier (to change state) application should lead to new minimal state whose elements' intersection with previous ones is not complete (at least some new boxes are introduced and some previous ones removed).
- Application of the same modifier twice should be unsuccessful.
- Application of invalid modifier (inconsistent with the previous ones) should be unsuccessful.
- Application of a valid modifier after rollback should be successful.
- Invalid modifiers should not be able to be added to history.
- Once an invalid modifier is appended to history, then history should not contain it and neither should it be available in history by it's id.
- History should contain valid modifier and report if a modifier if semantically valid after successfully appending it to history.
- BlockchainSanity test that combines all this test.

Coming test scenarios:
- Block application and rollback leads to the same history (rollback is not defined for history yet)
- NodeView apply block to both state and history or don't apply to any of them
- It is not possible to apply transaction to a state twice
- Tests for invalid transactions/blocks/etc

\knote{generators list}

\end{document}
