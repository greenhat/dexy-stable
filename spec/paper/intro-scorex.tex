% !TEX root = laws.tex

\subsection{The Scorex Framework}

The idea of a modular design for a cryptocurrency was first proposed by Goodman in the Tezos position paper~\cite{tezosPosition}. The paper (in Section 2) proposes to split a cryptocurrency design into the three protocols: network, transaction and consensus. In many cases, however, these layers are tightly coupled and it is hard to describe them separately. For example, in a proof-of-stake cryptocurrency a balance sheet, which representation is heavily influenced by a transaction format, is used in a consensus protocol. 

Plenty of modular open-source frameworks were proposed for speeding up development of new blockchain systems, including: Sawtooth~\cite{sawtooth} and Fabric~\cite{fabric} by Hyperledger, Exonum~\cite{exonum} by Bitfury Group, and Scorex~\cite{scorex} by IOHK. We have chosen Scorex, because it has finer granularity. In particular, in order to support hybrid blockchains as well as more complicated linking structures than a chain~(such as Spectre\cite{spectre}), Scorex does not have the notion of blockchain as a core abstraction. Instead, it provides a more general abstract interface to a \textit{history} which contains \textit{persistent modifiers}\footnote{In a blockchain-based cryptocurrency, the blockchain can be seen as the history and every block can be seen as a persistent modifier.}. The history is a part of a \textit{node view}, which is a quadruple of $\langle$\textit{history}, \textit{minimal state}, \textit{vault}, \textit{memory pool}$\rangle$. The node view is updated whenever a persistent modifier or a transaction is processed. The minimal state is a data structure and a corresponding interface providing the ability to check the validity of an arbitrary persistent modifier for the current moment of time with the same result for all the nodes in the network having the same history. The minimal state is to be obtained deterministically from an initial pre-historical state and the history. The vault holds node-specific information, such as a user's wallet. The memory pool holds unconfirmed transactions being propagated across the network by nodes before their inclusion into the history. Such a design, described in details in Section~\ref{sec:scorex}, gives us the possibility to develop an abstract testing framework where it is possible to state contracts for the node view quadruple components without knowing details of their implementations.